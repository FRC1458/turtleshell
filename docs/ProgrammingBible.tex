\documentclass[]{report}
\usepackage{textcomp}
\usepackage{graphicx}
\usepackage{hyperref}
\usepackage{listings}

%opening
\title{Programming Bible with TurtleShell}
\author{Sweeney}
\date{2016}

\begin{document}
\newcommand{\ITwoC}{I$^{2}$C}
\newcommand{\trademark}{$^{TM}$}
\newcommand{\goodcopyright}{$^{\copyright}$}
\newcommand{\pokemon}{Pok\'emon}
\maketitle

\tableofcontents

\chapter{Introduction}
Hello.

\section{Welcome to Programming!}
Welcome, and congratulations on your decision to learn more about programming!
Programming is the best group within a robotics team, and you should be proud to be a member.

\section{A Programming Lead}
The Programming Lead is the leader of programming team.
They are ultimately responsible for ensuring that the code works and fulfils its purpose.

The Responsibilities of the Programming Lead
\begin{itemize}
\item Design and implement the robot code.
\item Teach younger programmers and pass on the knowledge.
\item Work with drive team to design and implement a control system.
\end{itemize}

\section{Programmer Etiquette}
This section covers the standard programmer and behaviour at Team 1458, which may or may not be applicable to other teams.
\subsection{Cookies}
Programmers have a well-known sweet tooth, in particular for cookies.
The best type of cookies are chocolate chip cookies, with the subtype of chocolate chip with M\&Ms being the only improvement.
"Cookies" with nuts, raisins, or oatmeal are not real cookies, they should be safely disposed of in the trash.
Programmers are known to consume copious amounts of cookies in one session, particularly during build season.
\subsection{Turtwig and Turtles}
Turtwig, a \pokemon{} with the National Pok\'edex Number 387, is the cutest and the best \pokemon.
Turtwig is a turtle, and by association turtles are loved and appreciated as well.
Numerous Turtwigs are brought in during build season, and a Turtwig should be present during every major test.



\chapter{Physical Hardware}
	This section covers the physical devices associated with the robot that must be understood in order to properly program the robot.
	A basic familiarity with all parts of the robot is required, but certain parts, namely those associated with the control system must be understood in greater depth.
	This section will dedicate time to each class of components in rough proportion to their importance to programming.
\section{A Note On The Separation of Programming and Electrical}
	The boundary between programming and electrical, especially with sensors, can be unclear.
	The generally accepted split is that electrical is in charge (no pun intended) of the hardware and wiring, while programming is responsible for the control logic.
	However, programming is often given control over some aspects of hardware for the placement of sensors, although it is still electrical'€™s responsibility to wire the sensors.
\section{Motor Controllers}
	Motor controllers are what regulate the motors.
	They have a connection to the Power Distribution Panel (PDP), the motors themselves, and the roboRio through a PWM (or Pulse Width Modulation) control.
	They take the PWM signal and use it to control the voltage going to the motor, and thus control the motor.
\subsection{Victor 88x}
	Victor 884 and 888s are the simplest type of motor controller, and overall they are effective.
	The differences between the two are minor, with some changes to the PWM curve.
	They are the second-largest motor controller, with the required fan taking up much of the space.
	They have been made obsolete by the introduction of the Victor SP, which is strictly better.
\subsection{Victor SP}
	The Victor SP is VEX's new motor controller, and features several improvements over the previous models, including a much smaller form factor and an integrated heatsink which leaves a fan optional.
	Use-wise, it functions similar to the previous models with a PWM input.
\subsection{Talon SR}
	The Talon SR is very similar to the Victor SP, although it has a larger footprint more comparable to the Victor 88x models.
	It has been deprecated in favour of the Talon SRX, but the Victor SP is a more direct successor in function.
\subsection{Talon SRX}
	The Talon SRX is a much more advanced motor controller, comparable to a Jaguar in power.
	It can connect over CAN or PWM, and features functions that make it easier to set up PID control.
	However, this does lead to a more expensive motor controller and reduced programmer control over functionality.
\subsection{Jaguar}
	Jaguars are the most advanced motor controller, but they suffer heavily for their features, which include connectiblity over CAN, PWM, and Ethernet.
	Jaguars are far larger than other motor controllers, in footprint, height, and cost.
	They suffer an even larger lack of programmer control, with them refusing to work if they are at risk of overheating.
\subsection{Note on Programmer Control}
	Best practices developed at Team 1458 have been to do the majority of the coding manually.
	There have been issues with documentation and errors in WPILib, leading to the restriction of its use to the component interactions.
	Not only does coding it ourselves allow us to avoid these pitfalls, but allows programmers to get additional experience and learn more.
	Thus, WPILib is only used in TurtleShell for motor \& sensor interfaces, no higher-order logic is taken from it.
\section{Chassis/Drive Train}
The chassis/drivetrain refers to the physical structure of the robot. This is the domain of mechanical, and programmers simply need to understand how the motor movements correspond to actions in the physical world. More information on drive systems can be found in Part 3.

\section{Electrical Devices}
	This section covers the Power Distribution Panel (PDP), Pneumatic Control Module (PCM), and the Voltage Regulator Module (VRM).
	These devices are usually not used directly by programming, but understanding their function is necessary.
\subsection{Power Distribution Panel (PDP)}
	The PDP is what provides power to most of the robot.
	It has a connection to the battery through the circuit breaker, which serves as the on/off switch on the robot.
	It then connects through circuit breakers to all devices on the robot that require power.
	It is equipped with CAN in order to monitor the connections and their current draw.
\subsection{Pneumatic Control Module (PCM)}
	The PCM is connected over CAN to the roboRIO.
	All controls for pneumatic devices go through here.
	It provides power for the compressor and regulates it, along with controlling any solenoids.
\subsection{Voltage Regulator Module (VRM)}
	The VRM is connected to the PDP, and provides lower-voltage power.
	It provides power to the router on board, as well as other devices that don't require high amperages (so not most motors), such as LED lights.
\section{Sensors}
	This section discusses a variety of types of different sensors available on the robot, and their possible uses and drawbacks.
	It doesn't cover their coding or integration into software.
	This section is divided based on utility, however some sensors may fit under multiple categories.
\subsection{Distance}
	Distance sensors determine how far away something is from the sensor.
	It can be useful in aligning the robot or in preventing collisions.
	\subsubsection{Infrared}
		Infrared sensors detect how far away something is from the sensor based on the timing of pulses of infrared light.
		They are able to detect small objects and work in enclosed spaces, but are more expensive than ultrasonic sensors.
	\subsubsection{Ultrasonic}
		Ultrasonic sensors emit sounds above the range of human hearing and count the time it takes the sound to return.
		They are cheap, however they are bad at detecting small or curved objects, and are subject to interference from objects in the way.
	\subsubsection{Laser}
		Laser distance sensors are extremely accurate, quite often on the millimetre level, and do not suffer from as much interference as other sensors. However, they are the most expensive and fragile, qualities often unsuitable for the robot.
\subsection{Rotation}
	\subsubsection{Gyroscope}
		Gyroscopes are the best known rotation sensors.
		They are actually unable to measure the way the robot is facing, instead it measures the change and uses that to determine the direction relative to the start.
		They are widely available, but they have a problem in accuracy called yaw drift, where the angle will shift by 2-3 degrees per minute, even while stationary for the best gyros, and larger drift for lower quality.
\subsubsection{Magnetometer}
		Magnetometers measure the strength of magnetic fields on certain axes.
		In the context of the FRC, they are used to measure the Earth'€™s magnetic field and thus act as a compass, providing a reference point for rotation.
		While they don't have the same problem with drift, they are much more difficult to calibrate, with the calibration changing based on where the robot is located and what ambient magnetic fields exist.
		This can prove a problem, as the robot's motors generate magnetic fields which can interfere with proper function.
		They are also much more difficult to program.
\subsection{Vision}
	These sensors work with vision, which can be used for a variety of tasks depending on the year.
\subsubsection{Camera}
		Cameras are really the only device that can be used for vision, the lights are only there to assist the camera.
		The camera can see the retroreflective targets that are often visible during the autonomous period, and so the robot can take action based on that.
		Coding the camera is a difficult task, several libraries such as GRIP and NIVision are able to work with it.
\subsubsection{Light}
		The lights are to illuminate the retroreflective target so the camera can identify it.
		They are usually green, as green stands out the most from other colours present.
\subsection{Contact}
	\subsubsection{Pushbutton}
		Pushbutton switches consist of a button that makes (or sometimes breaks) an electrical connection.
		They can be useful as bumpers to know if the robot has made contact.
	\subsubsection{Limit Switch}
		Limit switches have some sort of object that when it is pressed down, it triggers.
		They are electronically identical to pushbuttons, but they have a switch rather than a button.
		They are useful in safety for moving devices, to ensure they don't go too far and damage something.
\subsection{Position}
	\subsubsection{Accelerometer}
		Accelerometers don't actually measure position, they measure acceleration.
		It'€™s technically possible to derive (well, integrate) position from acceleration, but it is a difficult process.
		Accelerometers can still be used in autonomous for positioning, but aren'€™t very useful beyond that.
		In years where the robot is tilting, it can be used to determine whether or not the robot is flat.
	\subsubsection{Rotary Encoder}
		Rotary encoders (usually just called encoders) are one of the most useful sensors.
		They record rotations of a shaft, which can be used to do everything from measuring the distance the robot has moved forward to the rotation of an appendage.
		They work by causing an electrical pulse whenever the shaft moves a certain amount (Such as 1\textdegree).
		The roboRio counts these pulses and uses that to determine how far the shaft has rotated.
		In order so that to roboRio can determine which direction the shaft is rotating, there are two channels that are slightly offset, so the roboRio can tell the direction by which triggers first.
		This means that it requires two digital inputs.
\section{roboRio}
	The roboRio is the brain of the robot.
	The code runs on the roboRio, which runs a modified version of Linux.
	The roboRio is one of the few "smart" devices on the robot, with most of the remaining devices serving to interface between it and the physical world.
	It has digital, analog, \ITwoC, SPI, USB, CAN and Ethernet I/O (Input/Output), along with PWM output and the MXP. (Don't worry, all of those acronyms will be explained shortly)
	By the FRC rules, all control of motors has be done through the roboRio.
	
\subsection{MXP (MyRio eXpansion Port)}
	The MXP is the extra set of pins in the middle of the roboRio.
	It is meant for custom devices, although a few commercial ones are available.
	It has a massive amount of possible inputs, about doubling the possible amounts of each input and output.
	Its use is largely unnecessary unless a large amount of inputs or outputs are needed.

\subsection{Digital I/O}
	The digital input and output ports use the same cables as the PWM inputs.
	The red and black are to power the sensors (with red power and black ground), while the white carries the signal in a digital fashion.
	Digital is where there are only two values, on and off.
	Sensors that use this include limit switches and rotary encoders.

\subsection{Analog Input}
	Analog Inputs are similar to the digital inputs, with the same cables and colours for power, ground, and signal.
	Analog is where the voltage varies, so there is a continuous range of values.
	Analog inputs power different kinds of sensors, with some gyroscopes and accelerometers, and most infrared and ultrasonic sensors relying on an analog input. The voltage from these must be interpreted to get meaningful results.

\subsection{PWM}
	PWM (Pulse Width Modulation) is how the roboRio communicates with the motor controllers.
	PWM is a system that changes the width of pulses, usually in the range of fractions of milliseconds, to alter motor power or servo position.
	It uses the same cables as analog and digital inputs.
	The main use of PWM outputs is to control motors.

\subsection{\ITwoC}
	\ITwoC (Pronounced Eye-two-see) is another communication protocol.
	It is very advanced, and is capable of exchanging large amounts of information and connecting multiple devices, but is very complicated (An interviewed company contracted it out rather than work with it).
	Luckily, some of the very low-level communication is done with the libraries, so the programmer only€ has to work with the bytes that are being transferred, rather than the process of transferring them.
\subsection{SPI}
	SPI is another complicated communications protocol, one which is not commonly used in FRC.
	Wikipedia states:
	\begin{quotation}
	The Serial Peripheral Interface (SPI) bus is a synchronous serial communication interface specification used for short distance communication, primarily in embedded systems.
	The interface was developed by Motorola and has become a de facto standard.
	Typical applications include sensors, Secure Digital cards, and liquid crystal displays.
	SPI devices communicate in full duplex mode using a master-slave architecture with a single master.
	The master device originates the frame for reading and writing. Multiple slave devices are supported through selection with individual slave select (SS) lines.
	Sometimes SPI is called a four-wire serial bus, contrasting with three-, two-, and one-wire serial buses.
	The SPI may be accurately described as a synchronous serial interface, but it is different from the Synchronous Serial Interface (SSI) protocol, which is also a four-wire synchronous serial communication protocol, but employs differential signalling and provides only a single simplex communication channel.
\end{quotation}
\subsection{CANbus}
The CAN (controller area network) bus is another way of interfacing with devices.
It is commonly used in industry, and it is found in modern cars.
It involves the high, yellow wire and the low, green wire.
They can be used for Jaguars, and other advanced electronics.
One key aspect is their use in of pneumatics and the PDP,  CAN is the only way to connect to those.
All workings with the CAN system are hidden in WPILib, a reasonable approach given its complexity.
The PDP and respective pneumatic classes can be used to interact with those systems.

\subsection{Ethernet}
There is an Ethernet port on the roboRIO.
Ethernet is a standard for communication, often used as a wired way to obtain internet access.
For the robot, the Ethernet port is used to connect to the router that provides wireless connectivity.
Cat 5e or above is recommended, as well remember to make sure there are not significant kinks in the cable, it degrades performance.

\subsection{USB}
	There are three USB ports on the roboRIO.
	Two are USB Type-A, the standard USB port.
	Devices such as cameras should be connected here, as well as any sensors that work over USB.
	\newline
	The USB Type-B (The cable with the pentagonal other end, often used for printers) port is useful for connecting to the computer, use it instead of Ethernet for direct connections to the roboRIO.

\section{Radio}
	The radio is used on the robot to communicate with the FMS and Driver Station. It is a router running specialised software.
	The model is OM5P-AN.
	It has a power plug which connects to the VRM, as well as two Ethernet ports, one for connection to the roboRIO and one for connection to the computer.
	 The Radio Configuration Utility, included with LabView, will install and flash the firmware.
	 Note that the configuration for home and competition use are different.

\section{The Human Body Analogy}
	For many people, an analogy comparing the parts of the robot to parts of the human body is useful for understanding the way the robot works and how things interact.
	Henceforth, let the analogy commence:
\subsection*{Skeletal System}
The frame and mechanical portions of the robot.
\subsection*{Muscular System}
Motors and solenoids.
\subsection*{Circulatory System}
Components carrying electricity for the purpose of powering components (not signals).
\subsubsection*{Heart}
The PDP.
\subsubsection*{Arteries}
Red cables, where power flows out.
\subsubsection*{Veins}
Black cables, where power flows back.
\subsubsection*{Lungs}
The battery, the source of all electrical power on the robot.
\subsection*{Nervous System}
"Smart" components as well as signal-carrying wires.
\subsubsection*{Axons}
Signal-carrying wires, most colours but red and black, especially white, yellow, and green.
\subsubsection*{Nerve Endings}
Sensors.
\subsubsection*{Brain}
The roboRIO, other computing devices would be secondary brains.



\chapter{Programming on the roboRIO}
This chapter will cover the roboRIO, languages running on it, the Java language,
and some basic Java code on the robot.
This chapter will not cover TurtleShell, and some things introduced in \ref{SimpleJavaontheroboRIO} will be overridden in \ref{TurtleShell}.


\section{roboRIO Basics}
This section covers basic roboRIO functionality from a programming perspective.

\subsection{Basic Specifications}
The roboRIO is a proper computer.
It is produced by National Instruments, based on the Xilinix\goodcopyright{} Zynq\trademark{}-7020 platform.
It has as its processor an ARM Cortex\trademark{}-A9, with 256MB of RAM and 512MB of flash storage.
It runs a real-time variant of Linux, and supports LabVIEW, C++, and Java.
This may be a restatement of the fact sheet.

\subsection{Connecting}
It is possible to connect to the roboRIO over both Ethernet and USB.
Either is acceptable, Ethernet should have a higher throughput but this is often not the case.
Whichever is more accessible (USB Type-B or Ethernet) is likely simpler to use.
The standard mDNS address is "roboRIO-\#\#\#\#-FRC.local", where \#\#\#\# is the team number. (No leading zeroes)

\subsubsection{mDNS}
The mDNS protocol is used for robot connection, reducing the need for static IP and other network setups.
It allows the use of host names rather than IP addresses on small networks without the need for a nameserver (such as Google's 8.8.8.8 and 8.8.4.4).
In order for computers to make use of it, they require external software, Apple's Bonjour works on Windows and Mac, and LabView comes with its own version.
Linux mDNS software also exists.

\subsubsection{Browser Use}
For some ungodly reason, NI decided to use Microsoft Silverlight to create the web interface for the roboRIO.
This restricts browser use to browsers which still allow that mess of a plugin, and NI happily recommends IE.
Team 1458 uses Firefox for this purpose.

\subsubsection{USB}
The driver for the roboRIO should automatically install. If not, it should install with LabView.
The mDNS address should work, alternatively the USB IP is "172.22.11.2".
Again note the USB port that should be used is the USB Type B port, often found on printers.
It is a house-looking hexagon.

\subsubsection{Ethernet}
The mDNS address can be used over Ethernet as well.
The radio uses the IP address 10.TE.AM.1 (Where TE.AM is the zero-padded team number), and gives other devices IP addresses in the range 10.TE.AM.10 to 10.TE.AM.99 through DHCP.

\subsection{Updating roboRIO Programs}
The roboRIO's firmware and image may need to be updated.
The firmware rarely needs to be updated, unless the device is being wiped, but the image is usually updated once per year.

\subsubsection{Updating the Firmware}
\begin{enumerate}
\item Connect to the roboRIO using USB, \textbf{not Ethernet}.
\item Connect to the web interface and log in with the Username "admin" and a blank password.
\item Click the update firmware button, and navigate to the location where the firmware is stored
(C:\textbackslash Program Files (x86)\textbackslash National Instruments\textbackslash Shared\textbackslash Firmware\textbackslash cRIO\textbackslash 76F2 on Windows 64-bit),
and select the most recent firmware (should be the numerically largest version).
\item Click the begin update button and let it complete the update.
\end{enumerate}

\subsubsection{Updating the Image}
\begin{enumerate}
\item Connect to the roboRIO using USB, \textbf{not Ethernet}.
\item Launch the imaging tool, located at C:\textbackslash Program Files (x86)\textbackslash National Instruments\textbackslash LabVIEW 2015\textbackslash project\textbackslash roboRIO Tool\textbackslash roboRIO\textunderscore ImagingTool.exe for 64-bit Windows.
\item Scan for roboRIOs and select the appropriate one.
\item Enter your Team Number in the box.
\item Check the box Console Out under Startup Settings and \textbf{uncheck} the box Disable RT Startup App.
\item Check the Format Target box and select the most recent image.
\item Click Reformat.
\item After the reformatting, press the reset button on the roboRIO to complete the process.
\end{enumerate}

\subsubsection{Installing Java on the roboRIO}
This step is only necessary to run Java programs.
\begin{enumerate}
\item Java 8 must be installed on the computer used to connect to the roboRIO.
\item Connect to the roboRIO using USB, \textbf{not Ethernet}.
\item Open the FRC Java Installer, located at "~/wpilib/tools/java-installer.jar", where ~ is the home directory.
\item Follow the listed instructions in the Java Installer.
\item Java 8 is now installed on the roboRIO. This will be needed to be repeated each time the roboRIO is imaged.
\end{enumerate}

\subsection{The roboRIO Web Interface}
The web interface allows use and customisation of many features on the roboRIO.
Many of these features require logging in, the default login is "admin" with no password.
However, a password must be set to access the file browser.
It is a good idea to create a second account with the same permissions and a password so that the file browser can be used, and the admin account is still available with the default settings.
For changing files in the lvuser account, the lvuser account must be logged in to, it has the password "password".
Other than the file browser, the main configuration tool is the permission manager.
Other than creating new accounts, the use and function is relatively complex and unimportant.
Although there are a variety of tabs, most of them are not important.
However, the main tab has the ability to update the firmware for the roboRIO, PCM, and PDP.


\section{roboRIO Intermediate and Advanced}
This section covers more advanced topics with the roboRIO such as the command line and processing capabilities.

\subsection{Moving Beyond the Web Interface}
While the web interface is easy to use, it is quite clunky and is often less intuitive than other options.
Most of these are command line based, such as scp and and ssh, and are from Unix.

\subsubsection{ssh}
One of the most powerful tools is ssh, the remote shell program.
It allows logins over a network connection, giving full access to the command line.
PuTTY is a Windows program available that can fulfil this function.
The command line is a fairly standard Linux environment, but one important difference is the lack of a sudo or su command.
Thus, in order to modify the code or output files, the "lvuser" account usually must be logged into.

\subsubsection{scp}
Another tool available is scp, which stands for secure copy.
It allows the copying of files to and from the roboRIO, the ANT script that compiles the code uses scp to move the code onto the roboRIO.
As a standard Unix command, documentation is available elsewhere.

\subsection{Processing Capabilities and Limitations}
The roboRIO is a moderately powerful computer.
It is less powerful than a smartphone, but is comparable to portable game systems like the Nintendo 3DS.
This is relevant from a programmer's perspective because it limits what a programmer can do.
For example, if several million floating point calculations need to be done in order to execute a single motor movement, the code will be far too inefficient to be effective.
However, eking every last bit of efficiency out of code requires destroying any pretence of readability or maintainability.
For the roboRIO, and for modern programming on non-embedded systems in general, the balance is usually to simply neither do wildly wasteful things nor condense the code.
However, this balance shifts in locations like loops. There, since the code must be repeatedly, it is usually beneficial to optimise the code at the expense of ease of use.
For example, declaring a variable that would normally be declared inside a loop outside instead adds in an extra variable to work with, but it stops that variable from having to be allocated every time that code is run, and as such saves processing time.
A key field where this is an issue is vision code.
With hundreds of thousands of pixels, a slight mistake can greatly increase the amount of time required.

However, even optimally designed vision programs are quite possibly too much for the roboRIO.
Thus, other computers can be used on the robot, such as Nvidia's Jetson TK1, to supplement it and offload vision processing, not only speeding up the processing but ensuring that the vision processing cannot hog time from more important tasks.


\section{Languages}
Many languages are supported on the roboRIO.
The officially supported languages are Java, C++, and LabVIEW, although unofficial libraries for other languages such as Python are available.
This guide primarily covers Java, although this section will attempt to provide a brief overview of the languages and aid in the selection of the best language for the team.

\subsection{Java}
Java is a C-like language developed at Sun Microsystems and currently maintained by Oracle.
Java's main feature is its cross-platform support, available because of the JVM, or Java Virtual Machine, which runs the code, rather than the operating system itself.
This additional layer of abstraction comes with a cost however, Java code is almost always less efficient than code written in C or C++.
Java is quite often used for commercial applications, the cross platform nature reduces development costs and the relative ease with which it can be used make it quite popular as a server-side language, although it is being supplanted by Ruby and Python in many of these applications.
As well, Java is a part of the Advanced Placement Computer Science A course, which many students enrolled in FIRST may take, and can serve as a breeding ground for programmers.

On the robot, virtually all Java features are available, and the additional overhead is only an issue in vision processing.
Plenty of libraries and sample code are available for Java on the roboRIO.

\subsection{C++}
C++ is an industry-standard language that has stood the test of time.
It is an expansion on the earlier C language, adding object-oriented programming.
Most native programs are written with C++ or similar, likely including the program being used to view this file.
C++ can be difficult to work with because of its use of pointers, but the difficulty increase compared to Java is relatively minor.
C++ and other lower-level languages run more quickly than languages like Java or Python.

C++ is fully supported on the roboRIO and is a fine choice for robot development.
Plenty of libraries and sample code are available for C++ on the roboRIO.

\subsection{LabVIEW}
LabVIEW is National Instruments's graphical programming language.
It is primarily used for working with NI's products in robotics and research.
The language is visual, rather than text, based, and is designed with blocks to represent operation and wires for data flow.
The language suffers from extremely long compile times, often stretching into multiple minutes, and poorer performance than Java.
It also suffers from limited applicability, NI's products are the only use of the language and while the visual aspect may be appealing to beginners, it is drastically unlike other programming languages and so skills are often not transferable.
It can also be very difficult to debug, and the tangled mess the wires form is infamous.

LabVIEW is heavily optimised for the roboRIO, and includes utilities to ease development.
NI provides support and software for the product.

\subsection{Direct Comparisons}

Overall, C++ and Java are relatively equal.
Although C++ is more difficult to use, it makes up for this in performance.
However, Java is more often used in server-based technology, and as such can be more useful in the career field.
Java is also commonly taught with the Advanced Placement Computer Science A class, which can serve as a valuable source of programming experience.
The choice between these languages should depend largely on the particular team, and the experience they have.
While both languages have minor advantages over the other, mentor, student, school or sponsor experience can override this advantage and should be the deciding factor.

LabVIEW is overall inferior for robot development.
It has limited applicability outside of FRC and lacks the massive repository of libraries that other languages have.
Much of the skills and code are not transferable, making the learning wasted.
It is also more difficult to develop and debug.

\begin{table}[h!]
	\begin{tabular}{c|l|l|l|}
		& Java & C++ & LabVIEW \\ \hline \hline
		Performance & 3 & 5 & 1 \\ \hline
		Power & 5 & 5 & 3 \\ \hline
		Support & 4 & 4 & 3 \\ \hline
		Learning Curve & 3 & 1 & 2 \\ \hline
		Ease of Use & 4 & 3 & 2 \\ \hline
		Career Utility & 4 & 3 & 2 \\ \hline
		Overall Recommendation & 5 & 4 & 2 \\ \hline
	\end{tabular}
	\caption{Side by side comparison of languages on a 1-5 scale, with 1 being the worst and 5 being the best}
\end{table}

\section{Basic Java}
This section covers the basics of the Java programming language which are necessary for roboRIO programming.
These basics are approximately covered by the AP Computer Science A (AP Java) class, those enrolled in the class at the same time as their participation in programming should find they have the knowledge necessary.

\subsection{Resources for Learning Java}
A full introduction to Java is outside the scope of this guide, it will only be able to serve as a quick reference.
Resources for learning Java include:
\begin{itemize}
\item AP Java Barron's Book (\url{http://www.amazon.com/Barrons-AP-Computer-Science-7th/dp/1438005946})
\item Oracle (\url{https://docs.oracle.com/javase/tutorial/java/})
\item Codecademy (\url{https://www.codecademy.com/learn/learn-java})
\end{itemize}

\section{Simple Java on the roboRIO}
\label{SimpleJavaontheroboRIO}
This section examines basic programming using Java on the roboRIO.
Some information introduced here will be overridden in \ref{TurtleShell}.

\subsection{Sample, Command-Based, and Iterative Robot}
\label{samplecommanditerative}
In order for the robot code to run, a class in the package org.usfirst.frc.team\#\#\#\#.robot.Robot.java must exist (Where \#\#\#\# is the team number) and extend RobotBase in some fashion.
To do this, WPILib provides SampleRobot.java, CommandBasedRobot.java, and IterativeRobot.java, and most robot code should extend one of these three classes.
Each of them has particular advantages and disadvantages.

\subsubsection{Sample Robot}
Sample Robot, previously known as Simple Robot, is quite simple, and allows the most control over program flow.
It features a method that is called \textbf{once} at the beginning of each period.
For example, the method that all teleoperated is contained in is:
\begin{lstlisting}
public void teleOperated() { ... }
\end{lstlisting}
Of note is that a loop must be contained within this method in order to get the teleoperated code to run more than once.
Sample Robot is easy to use for both very simple and very complex programs, but suffers for those of intermediate complexity.

\subsubsection{Command-Based Robot}
Command Based Robot is a very different paradigm than others.
It focuses on building complex functionality out of simpler functionality through "commands".
Team 1458 does not have any experience with it and as such cannot offer any advice.

\subsubsection{Iterative Robot}
Iterative Robot is similar to Sample Robot, but it externalises the loop.
Instead of a single method for each mode, an init method and a loop method are provided, with the init method called to initialise and the loop being called about every 20 ms.
It is a more efficient class than Sample Robot, because the Driver Station only sends information every 20 ms, and so it doesn't waste processing time. However, it does restrict flexibility in design, and can require global variables.

\section{Using Eclipse with WPILib}
This section covers the use of the Eclipse IDE.
Eclipse is one of the more popular IDEs, and is a FOSS project.


\subsection{Installation}
The installation instructions are available online at \url{https://wpilib.screenstepslive.com/s/4485/m/13503/l/145002-installing-eclipse-c-java}.

\subsection{Creating a Project}
\begin{enumerate}
\item Press the "New" Button.
\item Scroll down to "WPILib Robot Java Development".
\item Click the arrow to drop down the options below.
\item Select "Robot Java Project".
\item Click "Next".
\item Enter a project name.
\item Select a project type (See \ref{samplecommanditerative} for choosing a Project Type).
\item Press "Finish".
\end{enumerate}

\subsection{Creating code}
To create a class, do the following:

\begin{enumerate}
\item Right click on the project in the Package Explorer.
\item Navigate to "New".
\item Click "Class".
\end{enumerate}

\subsection{Importing a project}
\begin{enumerate}
\item Right click in the Package Explorer.
\item Select Import.
\item Under "General", select "Existing Projects into Workspace".
\item Click "Next".
\item Click on "Browse" next to "Select root directory".
\item Navigate to the directory containing the root directory of the project.
\item Click "Open".
\item Click "Finish".

The project is now imported, but for it to work properly, several additional steps must be taken.
\item Right click on the project in the Package Explorer.
\item Click on the "Properties" item.
\item Click on the "Java Build Path" item.
\item Click on the "Libraries" tab.
\item Click on the "Add External JARs..." button.
\item Navigate to \href{run:~/wpilib/java/current/java/lib}{~/wpilib/java/current/java/lib}.
\item Navigate to "~/wpilib/java/current/java/lib". (~ refers to the home directory, located at "/Users/username/" on Mac, and "C:\textbackslash{}Users\textbackslash{}username" on Windows, and "/home/username" on UNIX.)
\item Select all four items and click "Open".
\item Click "Apply".
\item Click on the "Source" tab.
\item Select all items in the view.
\item Click the "Remove" button.
\item Click on the "Add Folder..." button.
\item Select the "src" folder.
\item Click the "OK" button.
\item Click the "Apply" button.
\item Click the "OK" button.

\end{enumerate}
\section{Useful Classes and Methods in WPILib}
This section covers classes and methods that are needed in WPILib.

\chapter{TurtleShell}
\label{TurtleShell}
TurtleShell is the framework designed by Team 1458 to simplify best practices in robot coding as well as address issues within WPILib.
It is available at \url{https://github.com/FRC1458/turtleshell/}.
\section{Design Philosophy}
The most important idea of TurtleShell is that of modularity, and towards that, interfaces are used throughout.
The advantage of using modular components are that the code is more resuseable, and it can easily adapt to changes, such as when the motor controller is changed, it can be fixed in the code with a single line.


\section{Basic Functionality}
This section covers basic functionality of TurtleShell, and should allow a reader to successfully create basic robot code.

\subsection{Program Flow}
Similarly to WPILib, a Robot.class must exist, and it still extends RobotBase indirectly.
However, instead of defining its own components, the ObjectHolder interface is used to define and hold all robot components.
The Robot class calls the teleUpdateAll() method on the ObjectHolder in order to teleUpdate() all of the components.

The ObjectHolder can either be entirely custom written, or can extend SampleRobotObjectHolder, which defines an ArrayList to store the components and iterates over each element in the list and calls update on them when nessecary.
In the constructor of the ObjectHolder, it should declare all of the RobotComponents and pass in all data to properly initialise them.

The body of Robot.class should initialise an ObjectHolder, and in the teleOperated() method it should repeatedly call teleUpdateAll() on it.

\subsection{Key Interfaces}
This section covers key interfaces that must be understood to properly make use of TurtleShell.

\subsubsection{RobotComponent}
All components on the robot, from the chassis to an arm, implement this interface.
It provides the teleUpdate() and autoUpdate() methods.
These should all be contained in the ObjectHolder.

\subsubsection{ObjectHolder}
This interface is for containers for RobotComponents and similar.
It contains methods to call the teleUpdate() and autoUpdate() methods for its RobotComponents.

\subsubsection{Movement}
\paragraph{TurtleMotor}
An interface representing a motor that is given a MotorValue to move, representing full forward to full backwards.
\paragraph{TurtleServo}
An interface representing servos, which are given an angle and move to it.
\paragraph{TurtleSolenoid}
An interface representing a solenoid, which is either extended or retracted.

\subsubsection{Sensor}
\paragraph{TurtleButtonSensor}
An interface representing a sensor that is either pressed or not.
\paragraph{TurtleRotationSensor}
An interface representing a sensor that measures rotation around an axis.
\paragraph{TurtleDistanceSensor}
An interface representing a sensor that measures distance travelled by something.

\subsubsection{Unit}
This interface is used for classes the represent some type of measurement, such as distance, time, or angle.
All implementations wrap doubles, primarily for type safety.

\paragraph{Time}
This implements Unit, and represents time in seconds.
\paragraph{MotorValue}
This implements Unit, and represents a motor power, from -1 to 1.
\paragraph{Distance}
This implements Unit, and represents a distance in inches.
\paragraph{Angle}
This implements Unit, and represents an angle in degrees.
\paragraph{Rate}
This implements Unit. It is generified, so it represents another Unit per Time.

\subsubsection{Input}
\paragraph{TurtleDigitalInput}
TurtleDigitalInput represents an input which has discrete values.
\paragraph{TurtleAnalogInput}
TurtleAnalogInput represents an input which has continuous values.

\subsection{Key Implementations}
This section explains provided implementations for the above implementations and how they can be used.

\subsubsection{SampleRobotObjectHolder}
SampleRobotObjectHolder implements ObjectHolder, it is abstract and defines an ArrayList to store RobotComponents.
It also defines the teleUpdateAll() and autoUpdateAll() methods which simply iterate over the ArrayList and update each component.

\subsubsection{Movement}
All of the currently legal FRC motor controllers are implemented, with two separate implementations for the Talon SRX, one for PWM and one for CAN.
There is also a "FakeMotor" which emulates a motor controller but in fact does nothing.

\subsubsection{Sensors}
\paragraph{TurtleXtrinsicMagnetometer}
TurtleXtrinsicMagnetometer is a class for an Xtrinsic magnetometer.
It has 3 TurtleRotationSensors representing the pitch, yaw, and roll axes.
\paragraph{TurtleAnalogGyro}
TurtleAnalogGyro is a wrapper for WPILib's AnalogGyro class.

\subsubsection{Input}
\paragraph{TurtleJoystickAxis}
TurtleJoystickAxis implements TurtleAnalogInput, it takes a single axis from a Joystick and turns it into an AnalogInput with the range [-1,1].
\paragraph{TurtleJoystickButton}
TurtleJoystick Button implements TurtleJoystickButton.
It returns 1 if the button is pressed, and 0 otherwise.
\paragraph{TurtleJoystickPOVSwitch}

\section{Advanced Features}
pid


\section{Code Style}
This section covers the code style used throughout TurtleShell and the code of Team 1458.
\subsection{Naming}
The naming convention is very similar to the standard Java naming convention.
Classes begin with a capital letter, and camel case is used throughout, with the first letter lower case for instance variables.
Global constants should be in all capitals.

The prefix "Turtle" is used for most classes in TurtleShell.
This helps to avoid namespace collisions (having the same name for two different classes), and also serves to easily catergorise classes which are a part of the library.

\subsection{Comments}
The main use of comments should be for JavaDocs on publicly accessible classes, methods, and constructors.
They should detail the interactions with the object, but the object itself should be able to be considered as a black box.
Comments inside methods or other blocks of code are in general discouraged, because if they become out of sync with the code they add to confusion, but should be used for complicated or seemingly illogical methods or algorithms.

\subsection{Indenting, Spacing, and Braces}
The default Eclipse settings should be close to suitable for formatting, Control-Shift-F, Command-Shift-F, or Meta-Shift-F should automatically format the code correctly.
Tabs should be used, with a width of 4 spaces.
Braces should appear at the end of the line, not on a new line by themselves.
If a line of code is too unwieldy to fit on a single line, line breaks should take place at logical breaks, such as an "and" or an "or" statement.
Line Feeds should be used as a line terminator, there should be a trailing newline.

\end{document}
